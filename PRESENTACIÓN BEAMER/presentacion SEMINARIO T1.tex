\documentclass[10pt]{beamer}
\usetheme{CambridgeUS}
\usepackage[utf8]{inputenc}
\usepackage[spanish]{babel}
\usepackage{amsmath}
\usepackage{amsfonts}
\usepackage{amssymb}
\usepackage{graphicx}
\usepackage{ragged2e}
\author{Kevin García - Alejandro Vargas}
\title{Seminario: Tarea 1}
%\setbeamercovered{transparent} 
%\setbeamertemplate{navigation symbols}{} 
%\logo{} 
%\institute{} 
%\date{} 
%\subject{} 
\justifying
\begin{document}

\begin{frame}[plain]
\maketitle
\end{frame}

\begin{frame}{Contenido}
\tableofcontents
\end{frame}

\section{Antecedentes}
\subsection{Antecedente 1}
\begin{frame}
\frametitle{Planes de muestreo de aceptación de atributos mejorados basados en muestreo de nominaciones máximas}
Este artículo, de los autores Jozani \& Mirkamali (2010), demuestra el uso de la técnica de muestreo de nominación máxima (MNS) en diseño y evaluación de planes de muestreo de aceptación únicos para atributos.


~\\Se analiza el efecto del tamaño de la muestra y el número de aceptación en el rendimiento de los planes MNS propuestos mediante la curva característica de operación(OC). Entre otros resultados, se muestra que los planes de muestreo de aceptación de MNS con un tamaño de muestra más pequeño y un número de aceptación mayor se desempeñan mejor que los planes de muestreo de aceptación comúnmente usados para atributos basados en la técnica de muestreo aleatorio simple(SRS). De hecho, los planes de muestreo de aceptación MNS dan como resultado curvas de OC que, en comparación con sus contrapartes de SRS, están mucho más cerca de la curva de OC ideal.  
\end{frame}

\subsection{Antecedentes 2}
\begin{frame}
\frametitle{}
\end{frame}

\section{Efron \& Hastie (2017)}
\subsection{Tema 1}
\begin{frame}
\frametitle{}
\end{frame}

\subsection{Tema 2}
\begin{frame}
\frametitle{}
\end{frame}

\section{Lantz (2013)}
\subsection{Tema 1}
\begin{frame}
\frametitle{Aprendizaje probabilístico - clasificación usando bayes ingenuo(cap 4 - Lantz 2013)}
En este tema, se utiliza la estadística bayesiana, más específicamente el teorema de Bayes, para tratar de predecir o clasificar, modelando la incertidumbre por medio de probabilidades y utilizando internamente pruebas de hipótesis. 

~\\Es un método muy utilizado en la actualidad en diversos contextos por su uso práctico y los buenos resultados en la mayoría de los casos. Por ejemplo, se puede utilizar para algo muy trivial, como clasificar una persona en hombre o mujer basándonos en las características de sus medidas: peso, altura y número de pie; pero, también se puede usar para problemas más complejos y muy útiles como por ejemplo, el que plantea el autor, filtrar el spam de los teléfonos celulares para eliminarlos automáticamente y que esto no se convierta en una molestia para los usuarios.

~\\Con la proliferación de datos no estructurados, la clasificación de texto o la categorización de texto ha encontrado muchas aplicaciones en la clasificación de temas, análisis de sentimientos, identificación de autoría, detección de correo no deseado, etc. 
\end{frame}

\subsection{Tema 2}
\begin{frame}
\frametitle{Divide y vencerás - Clasificación usando reglas y arboles de decisión(cap 5 - Lantz 2013)}
La clasificación usando reglas y arboles de decisión, es un sistema de aprendizaje supervisado que aplica la estrategia ``divide y vencerás'' para hacer la clasificación, implementando métodos y técnicas para la realización de procesos inteligentes, representando así el conocimiento y el aprendizaje, con el propósito de automatizar tareas.

\textit{``Por su estructura son fáciles de comprender y analizar; su utilización cotidiana se puede dar en diagnósticos médicos, predicciones meteorológicas, controles de calidad, y otros problemas que necesiten de análisis de datos y toma de decisiones'', lo que quiere decir que los árboles de decisión pueden ser usados en cualquier ámbito sin importar que sea laboral o personal, siempre y cuando implique toma de decisiones con cierto grado de incertidumbre} (Calancha Zuniga, Carrión Bárcena, Cori Vargas, \& Villa Torres, 2010, pág. 2).

~\\Es evidente que obtener un mecanismo como este sería muy útil debido a que podríamos predecir comportamientos futuros a partir de los comportamientos observados en el pasado.
\end{frame}

\begin{frame}
  \frametitle{Bibliografía}
  
\nocite{A1, L, EyH}
  
  \bibliographystyle{plain}
  \bibliography{references}
\end{frame}
\end{document}
