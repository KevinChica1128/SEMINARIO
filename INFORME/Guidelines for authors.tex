%% -------------------------------------------------------------------------------------
% This is a general template file for preparing an article for Revista Colombiana de Estad�stica
% *************************************************************************************
\documentclass[english]{revcoles}
\usepackage[latin1]{inputenc}
%
\hyphenation{}
% //// --------------------------------------------------------------------------------
\begin{document}

\title[maintitle = Tarea 1,
]

\begin{authors}
\author[firstname = Kevin,
        surname = Garc�a,
        numberinstitution = 1,
        code= 1533173,
        affiliation = Universidad del Valle,
        email = kevin.chica@correounivalle.edu.co]
\author[firstname = Alejandro,
        surname = Vargas,
        numberinstitution = 1,
        code= 1525953,
        affiliation = Universidad del Valle,
        email = jose.alejandro.vargas@correounivalle.edu.co]
\end{authors}
%
\begin{institutions}
     \institute[subdivision = Departamento de estad�stica,
                institution = Universidad del Valle,
                city = Cali,
                country = Colombia]
\end{institutions}

\section{Antecedentes trabajo de grado}




\section{Efron \& Hastie (2017)}



\section{Lantz (2013) o Torgo (2017)}




\section{Citations}
Authors should use BiBTeX to prepare references whenever possible. Use the \emph{references.bib} style given in the \LaTeX\ macro package (see the website of the journal). References in the text are included by using  in other the commands \emph{cite} and \emph{citeasnoun}. Some examples of citations are

\begin{itemize}
\item \cite{Dodge-85},
\item \citeasnoun{Borges-05}
\item \citeasnoun{Conover-81}
\item \citeasnoun{R}
\end{itemize}

\section{Appendices}

Lengthy technical portions of a manuscript should appear in a separate appendix to the manuscript.


    \bibliography{references}

    \appendix% !don't modify this line�

\end{document}

